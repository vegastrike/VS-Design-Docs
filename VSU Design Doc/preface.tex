Welcome, intrepid reader, to the Vega Strike Universe - or at least a
document containing a lot of text concerning it and its design.

When we first started developing the current back story for VS, the
existing premise could be reasonably summed up as follows: ``An alien
species called the Aera are aggressively invading human space, and
there's this other group called the Rlaan who don't like either of the
humans or the Aera, but probably dislike the Aera more than the
humans'' - which is to say, someone out there had played StarCraft and
liked it. In the years since then, we've added a bit more, as the
somewhat larger size of this document may indicate. While this
document may take a bit longer to digest than the above one-sentence
summary, we happen to think it's worth the extra reading.

\section*{About this document}
This document is best described as a sort of ``design doc'' for the Vega
Strike Universe (hereafter often ``VSU'' -- this document uses a lot
of initialisms - please consult the \ref{Glossary:Glossary} section for any of them
that aren't clear). It contains a number of different things: general
cosmology and physics for the VS universe, back story for the {\it
Upon the Coldest Sea} game setting, discussions of design
philosophies, and discussions of the game-play-relevant implementations thereof. In that
it splits its time among sections discussing the design philosophy
behind the design of the VSU, sections best viewed as discussions of
the resulting decisions, and swaths of expository data dumps detailing
particulars of the backing fiction, it is an uneven document. However,
in serving all of the audiences it intends to inform, it would seem
difficult to achieve full blown elegance.

This document aims to do the following:
\begin{itemize}
\item Provide a cosmology and highlight key rules for the VS universe
   so that some intuition as to what is and is not likely to be canon
   can develop. This includes delineating where we are taking
   liberties with physics and where we are holding firm to our grip on
   reality. Moreover, this delineation must be addressed twice - once
   for the purposes of creating and editing fiction, and again for the
   artistic representations, which may not always correspond exactly
   to the ostensible VS reality (e.g. the VS universe demands that we
   have substantial radiators on starships during the {\it Upon the
   Coldest Sea} time period, but the artistic direction may resolve,
   on many craft, to token/symbolic radiator surfaces in order to preserve
   aesthetic and other artistic freedoms). Within the bounds of what
   has been declared possible in VS, we must likewise be sure to
   indicate where optimism is assumed and where the VS universe makes
   more pessimistic assumptions about how much of the possible is
   actually achievable by given groups at particular points in time.
\item Provide both back story and future-history for large-scale
  events, so that those looking to create stories set in the Vega
  Strike universe have both a backdrop to frame their stories in, and
  a set of future events to preemptively constrain continuity.
\item Codify game-play philosophies and examine the effect that universe decisions will have on potential game-play options, and vice versa.
\item Provide sufficient detail on relevant species, factions, and
  technologies, as they appear in the UtCS setting, that artists can
  become familiar with the subjects they are to portray.
\end{itemize}

Before going any further, an important point - at the time of this
writing, this document is far from finished. It is not polished, it is
not, in some places, even fully skeletal. This document will change
over time. The odds approach certainty that some pieces of it will
eventually require retconning and that others will be missed during various
evolutionary changes and thus become
desynchronized. Still, what is in this document, even in its
unfinished and mercurial state, is the VS canon until changed, and
should not be ignored, skipped over, or otherwise elided lightly.

\section*{Who is this document for?}
It should be acknowledged that this is a somewhat oddly targeted
document in there are several audiences that it needs to inform. These
audiences are rather diverse, ranging from texturing artists to
writers working on player-driven plot-lines. In keeping with this
mixed-usage model, a fair portion of the people who read this document
will not need to read the entire document. Artists, for example, are
encouraged to read the overviews in Chapter~\ref{chapt:overview}
before jumping to the portfolios portion of this document
(Chapter~\ref{chapt:portfolios}), but may find little use for much of
Chapter~\ref{chapt:timelines}'s treatment of VSU time-lines. That said,
those looking to do any additional fleshing out of the framework are
strongly-advised to read the whole document with some degree of
attention before considering pushing forward with such an undertaking.

\section*{Additional notes on reading this document}
Certain sections of this document may contain passages in the first
person. These can be assumed to be written by {\it JS} unless
otherwise specified or indicated by context. 

Sections not otherwise
specified should be assumed to be written from an omniscient
viewpoint. In particular, some of the Appendices, such as Appendix~\ref{appendix:Species} and Appendix~\ref{appendix:Factions} are {\bf NOT}
written from an omniscient viewpoint, and are intended to represent
generally available knowledge that would be easily obtained in the
UtCS time period. For information intended to player-visible, please
pay special attention to said appendices, {\it especially} when they
are in conflict with data from the omniscient viewpoint sections of
this text, as the difference is likely an intentional implementation
of ignorance or misconception on the part of the VSU's inhabitants.

Some parts may be so unpolished or incomplete as to be difficult to
parse or otherwise comprehend. Similarly, they may use non-standard
vocabulary or jargon that has yet to make it into the glossary or rely
on mention of external sources that have not yet been appropriately
added to the references section. However, these shortcomings are not
cause for ignoring the sections in question, but rather are cause for
developing questions pertaining to said sections, thereby prompting
their improvement. For insight, consider this previous VS-related
example of communication breakdown, one is told that ``X sounds like a
campanile'' and comes back with a sound for X that has no relation to
bell towers with an explanation along the lines of ``well, I didn't
know what a campanile was, so I just did something nice.'' The root
cause (campanile is apparently an insufficiently common word and
should be defined) is left untreated, the work in question balks
canonicity, and some fair bit of time may well have been spent on
something that could have been rectified with a simple clarifying
question. While asking the reader to invest additional energy in
comprehending the more tenuous portions of the text and actively
responding to their shortcomings is a more than normal demand,
attempting to use an incomplete document as a guide is a somewhat
awkward undertaking, and comes with its additional burdens. Those not
interested in walking through the textual debris and construction
zones are welcome to wait - we do eventually plan to make this
document fit for normal reading - but the realities of the schedules
we're working on means that this document will continue to see use in
various stages of its genesis, refinement, and extension.


% LocalWords:  UtCS Aera Rlaan StarCraft VSU initialisms retconning retconned
% LocalWords:  JS VSU's
